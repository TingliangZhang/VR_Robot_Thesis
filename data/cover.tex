\thusetup{
  %******************************
  % 注意:
  %   1. 配置里面不要出现空行
  %   2. 不需要的配置信息可以删除
  %******************************
  %
  %=====
  % 秘级
  %=====
  secretlevel={秘密},
  secretyear={10},
  %
  %=========
  % 中文信息
  %=========
  ctitle={基于Vive VR的遥操作机械手},
  cdegree={工学硕士},
  cdepartment={天空工场},
  cmajor={计算机科学与技术},
  cauthor={张庭梁},
  csupervisor={钟海旺老师},
  cassosupervisor={}, % 副指导老师
  ccosupervisor={}, % 联合指导老师
  % 日期自动使用当前时间,若需指定按如下方式修改:
  % cdate={超新星纪元},
  %
  % 博士后专有部分
  cfirstdiscipline={计算机科学与技术},
  cseconddiscipline={系统结构},
  postdoctordate={2009年7月——2011年7月},
  id={编号}, % 可以留空: id={},
  udc={UDC}, % 可以留空
  catalognumber={分类号}, % 可以留空
  %
  %=========
  % 英文信息
  %=========
  etitle={An Introduction to \LaTeX{} Thesis Template of Tsinghua University v\version},
  % 这块比较复杂,需要分情况讨论:
  % 1. 学术型硕士
  %    edegree:必须为Master of Arts或Master of Science(注意大小写)
  %             “哲学、文学、历史学、法学、教育学、艺术学门类,公共管理学科
  %              填写Master of Arts,其它填写Master of Science”
  %    emajor:“获得一级学科授权的学科填写一级学科名称,其它填写二级学科名称”
  % 2. 专业型硕士
  %    edegree:“填写专业学位英文名称全称”
  %    emajor:“工程硕士填写工程领域,其它专业学位不填写此项”
  % 3. 学术型博士
  %    edegree:Doctor of Philosophy(注意大小写)
  %    emajor:“获得一级学科授权的学科填写一级学科名称,其它填写二级学科名称”
  % 4. 专业型博士
  %    edegree:“填写专业学位英文名称全称”
  %    emajor:不填写此项
  edegree={Doctor of Engineering},
  emajor={Computer Science and Technology},
  eauthor={Zhang Tingliang},
  esupervisor={},
  eassosupervisor={},
  % 日期自动生成,若需指定按如下方式修改:
  % edate={December, 2005}
  %
  % 关键词用“英文逗号”分割
  ckeywords={机器人, 虚拟现实, 远程控制},
  ekeywords={\TeX, \LaTeX, CJK, template, thesis}
}

% 定义中英文摘要和关键字
% 
% 简化操作,降低成本,提高用户体验  实时性
% 
\begin{cabstract}
随着机器人技术的发展,机器人的应用场景越来越多。但是目前自动控制机器人尚不能执行多数复杂任务,特别是抢险救灾等需要随机应变的任务,这种情况下需要遥操作机器人。

目前机械臂远程控制普遍采用手柄或键盘控制方式,且监控方式普遍为摄像头图像显示在监视器上,与现场操作差别很大。我们开发了一套用VR设备远程控制机器人的系统以及配套的三维实时场景采集及图传系统,使操作者能有身临其境的操作体验,大幅降低成本的同时减小了延迟。

我们采用双目摄像头采集实时场景信息,两个目采集到的图像对应到VR眼镜的两个显示屏中,同时多自由度双目支架保证双目朝向和操作者双眼朝向一致,从而简单的使操作者可以看到实时的立体场景。

虚拟现实头盔和追踪器上的红外定位模块精确的确定了它们的绝对位置,由此可得它们的相对位置,我们可以使机械臂末端和双目摄像头的相对位置和其一致,从而实现操作者直接用手的位置来控制机械臂。

 \bigbreak
创新点及优势:
  \begin{itemize}
    \item 为使操作者看到具有立体感的实时画面,我们开发了由双目摄像头及VR显示系统组成的三维实时场景采集系统,此系统较图像拼接和场景重构实时性好,且对算力要求不高。
    \item 手持Vive追踪器操作机械臂末端符合我们日常使用手进行操作的习惯。较外骨骼和Optitrack运动捕捉系统廉价,且能够满足绝大部分需求。
  \end{itemize} 
  \bigbreak

经测试,本系统延迟很低,操作简单,且方便迁移至其他系统。

此系统可以帮助技术人员远程执行任务,而无需复杂的遥操作培训和练习。未来可以用于航天,拆弹,救援,深海作业,远程交互等领域。



\end{cabstract}

% 如果习惯关键字跟在摘要文字后面,可以用直接命令来设置,如下:
% \ckeywords{\TeX, \LaTeX, CJK, 模板, 论文}

\begin{eabstract}
   An abstract of a dissertation is a summary and extraction of research work
   and contributions. Included in an abstract should be description of research
   topic and research objective, brief introduction to methodology and research
   process, and summarization of conclusion and contributions of the
   research. An abstract should be characterized by independence and clarity and
   carry identical information with the dissertation. It should be such that the
   general idea and major contributions of the dissertation are conveyed without
   reading the dissertation.

   An abstract should be concise and to the point. It is a misunderstanding to
   make an abstract an outline of the dissertation and words ``the first
   chapter'', ``the second chapter'' and the like should be avoided in the
   abstract.

   Key words are terms used in a dissertation for indexing, reflecting core
   information of the dissertation. An abstract may contain a maximum of 5 key
   words, with semi-colons used in between to separate one another.
\end{eabstract}

% \ekeywords{\TeX, \LaTeX, CJK, template, thesis}
